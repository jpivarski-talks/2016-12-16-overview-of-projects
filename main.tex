\documentclass{beamer}

%
% Choose how your presentation looks.
%
% For more themes, color themes and font themes, see:
% http://deic.uab.es/~iblanes/beamer_gallery/index_by_theme.html
%
\mode<presentation>
{
  \usetheme{default}      % or try Darmstadt, Madrid, Warsaw, ...
  \usecolortheme{default} % or try albatross, beaver, crane, ...
  \usefonttheme{default}  % or try serif, structurebold, ...
  \setbeamertemplate{navigation symbols}{}
  \setbeamertemplate{caption}[numbered]
  \setbeamertemplate{footline}[page number]
  \setbeamercolor{frametitle}{fg=white}
  \setbeamercolor{footline}{fg=black}
} 

\usepackage[english]{babel}
\usepackage[utf8x]{inputenc}
\usepackage{tikz}
\usepackage{listings}
\usepackage{courier}

\xdefinecolor{darkblue}{rgb}{0.1,0.1,0.7}
\xdefinecolor{dianablue}{rgb}{0.18,0.24,0.31}
\definecolor{commentgreen}{rgb}{0,0.6,0}
\definecolor{stringmauve}{rgb}{0.58,0,0.82}

\lstset{ %
  backgroundcolor=\color{white},      % choose the background color
  basicstyle=\ttfamily\small,         % size of fonts used for the code
  breaklines=true,                    % automatic line breaking only at whitespace
  captionpos=b,                       % sets the caption-position to bottom
  commentstyle=\color{commentgreen},  % comment style
  escapeinside={\%*}{*)},             % if you want to add LaTeX within your code
  keywordstyle=\color{blue},          % keyword style
  stringstyle=\color{stringmauve},    % string literal style
  showstringspaces=false,
  showlines=true
}

\lstdefinelanguage{scala}{
  morekeywords={abstract,case,catch,class,def,%
    do,else,extends,false,final,finally,%
    for,if,implicit,import,match,mixin,%
    new,null,object,override,package,%
    private,protected,requires,return,sealed,%
    super,this,throw,trait,true,try,%
    type,val,var,while,with,yield},
  otherkeywords={=>,<-,<\%,<:,>:,\#,@},
  sensitive=true,
  morecomment=[l]{//},
  morecomment=[n]{/*}{*/},
  morestring=[b]",
  morestring=[b]',
  morestring=[b]"""
}

\title[2016-12-16-overview-of-projects]{Jim Pivarski's Overview of Projects}
\author{Jim Pivarski}
\institute{Princeton -- DIANA}
\date{December 16, 2016}

\begin{document}

\logo{\pgfputat{\pgfxy(0.11, 8)}{\pgfbox[right,base]{\tikz{\filldraw[fill=dianablue, draw=none] (0 cm, 0 cm) rectangle (50 cm, 1 cm);}}}\pgfputat{\pgfxy(0.11, -0.6)}{\pgfbox[right,base]{\tikz{\filldraw[fill=dianablue, draw=none] (0 cm, 0 cm) rectangle (50 cm, 1 cm);}\includegraphics[height=0.99 cm]{diana-hep-logo.png}\tikz{\filldraw[fill=dianablue, draw=none] (0 cm, 0 cm) rectangle (4.9 cm, 1 cm);}}}}

\begin{frame}
  \titlepage
\end{frame}

\logo{\pgfputat{\pgfxy(0.11, 8)}{\pgfbox[right,base]{\tikz{\filldraw[fill=dianablue, draw=none] (0 cm, 0 cm) rectangle (50 cm, 1 cm);}\includegraphics[height=1 cm]{diana-hep-logo.png}}}}

% Uncomment these lines for an automatically generated outline.
%\begin{frame}{Outline}
%  \tableofcontents
%\end{frame}

\begin{frame}{My goals}
\vspace{0.6 cm}
\mbox{\hspace{-0.6 cm}\begin{minipage}{1.1\linewidth}
\begin{enumerate}
\item \textcolor{darkblue}{To build bridges between the HEP software ecosystem and the big data ecosystems--- Scientific Python and Hadoop/Spark--- so that HEP data can easily flow between them.}

\begin{itemize}\setlength{\itemsep}{0.1 cm}
\item {\bf Scikit-HEP:} reorganize rootpy, root\_numpy, Ostap and maybe others into a Pythonic layer between HEP and Scientific Python.

\textcolor{gray}{\it with Eduardo, David, Noel Dawe, Vanya Belyaev, and Sasha Mazurov}

\item {\bf root4j:} pure-Java ROOT I/O for Spark integration.

\textcolor{gray}{\it with Viktor Khristenko for Oliver Gutsche and Matteo Cremonesi}

\item {\bf Scope:} NoSQL database/server for interactive analysis.

\textcolor{gray}{\it with Jin Chang and Igor Mandrichenko}
\end{itemize}

\item \textcolor{darkblue}{To build computational engines in those big data ecosystems that allow us to perform HEP-style analyses when we get there.}

\begin{itemize}\setlength{\itemsep}{0.1 cm}
\item {\bf Histogrammar:} functional interface to aggregation.

\textcolor{gray}{\it with Alexey Svyatkovskiy for Oliver Gutsche and Matteo Cremonesi}

\item {\bf Femtocode:} query language for Scope and Spark DataFrames.

\textcolor{gray}{\it just me}
\end{itemize}
\end{enumerate}
\end{minipage}}
\end{frame}

\end{document}
